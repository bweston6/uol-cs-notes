\documentclass{article}
\title{Tutorial for Week 6 - Answers}
\author{Ben Weston}
\date{\today}

\begin{document}
\maketitle
\begin{enumerate}\setcounter{enumi}{1}
        \item All of the following are solved by the pigeon-hole principle. Therefore the set $A$ must have a cardinality $\geq k|B|+1$ to satisfy the question.
                \begin{enumerate}
                        \item $A$ is the set of rolls and $B$ is the set of pairs of numbers: $$\{(1,1),(2,2),(3,3),(4,4),(5,5),(6,6)\},$$ such that the statement is satisfied. $|B|=6$, therefore $|A|=|B|+1=7$ rolls.
                        \item Two die can give any combination between 2 and 12. Therefore $|B|=11$. The number of rolls required is $|A|$, as $|A|=|B|+1$, $|A|=11+1=12$ rolls.
                        \item This is similar to the last question but $k$ is increased from 1 to 2, as there are now three matches required. This gives $k=n-1=3-1=2$. As the number of rolls required is $|A|$ and $|A|=k|B|+1$, $|A|=2\times11+1=23$.
                \end{enumerate}
\end{enumerate}
\end{document}
